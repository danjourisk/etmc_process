%% ETMC MEETING OPERATION PROCESS


%%%%%%%%%%%%%%%%%%%%%%%%%%%%%%%%%%%%%%%%%%%%%%%%%%%%%%%%%%%%%%%%%%%%%%%%%%
\section{Instruction}

This document specifies Ericsson Toastmasters Club(ETMC) regular meeting 
components and operation model, aiming at providing a reference model
that can be continuely improved by Ericsson Toastasters Club Officers, 
in order to keep advancing meeting quality, leading to better service to 
all ETMC members.

Furture revision to this document should happen on basis of a full 
understanding of Toastmasters International Competent Leadership(CL)
manual, Competent Communication(CC) manual and this document, under
the approval of all ETMC officers.

%%%%%%%%%%%%%%%%%%%%%%%%%%%%%%%%%%%%%%%%%%%%%%%%%%%%%%%%%%%%%%%%%%%%%%%%%%
\section{Terminology}

CC   Competence Communication
CL   Competence Leadership
ETMC Ericsson Toastmaster Club
VPE  Vice President Education
VPM  Vice President Membership
TTE  Table Topic Evaluator
TTM  Table Topic Master
TM   Toastmaster
NMC  New Member Ceremony

%%%%%%%%%%%%%%%%%%%%%%%%%%%%%%%%%%%%%%%%%%%%%%%%%%%%%%%%%%%%%%%%%%%%%%%%%%
\section{Meeting Role Definition}

An ETMC regular meeting has Toastmaster(TM), President, Vice President 
Membership, Vice President Education, Grammarian, Timer, Ah Counter,
Speaker, Evaluator, General Evaluator, Table Topicsmaster(TTM), Table 
Topic Evaluator(TTE) and Segent At Arm(SAA) roles to fully operate.  

\subsection{President}

President role in a regular meeting is usually focused on giving opening 
speech and closing speech, which act as an official way of information 
sharing and club promoting.

\subsection{Toastmaster}

Toastmaster is a genial host of regular meeting, who connects various 
session in a regular meeting by introducing participants on the stage 
and creating an atmosphere of interest, expectation, and receptivity.

\subsection{Vice President Education}

Vice President Education role do not usually appear in a regular meeting,
unless acting as a backup if President is absent. This role in a meeting
has the same function as President role does.

\subsection{Vice President Membership}

Vice President Membership(VPM) role appears only when there are new members 
who are ready for a New Member Ceremony(NMC) in a regular meeting. VPM will 
be the host for NMC.

\subsection{Grammarian}

Grammarian is responsible to introduce word of day in regular meeting to enlarge
members' vocabulary. This role also need to give comments on the use of English
during the course of the meeting.

\subsection{Timer}

Timer role is responsible for time control of regular meeting. Each role on
the stage is timed and showed different time signs as a reminder. Timer helps 
improve time management skills of members. 

\subsection{Ah Counter}

Ah Counter is responsible for noting words and sounds used as a crutch or 
pause filler by anyone on the stage during the meeting. Ah Counter can help 
reduce members' crutch words on the stage.

\subsection{Speaker}

Speaker role is the center of the whole regular meeting, who usually has an 
prepared speech, targeting at CC manual and other advanced speech manuals,
to practice communication and leadership skills.

\subsection{Evaluator}

Evaluator role focuses on evaluating the speaker's performance according to
specific speech project manual to help speaker achieve their goals. 

\subsection{General Evaluator}

General Evaluator gives evaluation in accordance to the conduction of the whole 
regular meeting, and all roles that do not have evaluator.

\subsection{Table Topic Master}

Table Topic Master role hosts table topic session in the meeting, which mainly
aims at providing members who doesn't have a registered role a chance to speak
at regular meeting, and practice their impromptu speech skills.

\subsection{Table Topic Evaluator}

Table Topic Evaluator role is responsible for evaluating the conduction of table
topic session by TTM, and other members performance at the stage.

\subsection{Segent At Arm}

Segent at Arm is responsible for collecting ballots and announcing the best at the
end of the meeting. 

%%%%%%%%%%%%%%%%%%%%%%%%%%%%%%%%%%%%%%%%%%%%%%%%%%%%%%%%%%%%%%%%%%%%%%%%%%
\section{Regular Meeting Agenda}

introduce each session


%%%%%%%%%%%%%%%%%%%%%%%%%%%%%%%%%%%%%%%%%%%%%%%%%%%%%%%%%%%%%%%%%%%%%%%%%%
\section{Meeting Role Standard Actions}

To guarateen an smooth operating regular meeting, each role taker shall 
at least complete the coresponding action related to a specific role. Meanwhile, 
role takers are encouraged to do more to improve meeting quality before, during 
and after the meeting.

\subsection{Toastmaster}

Toastmaster is the main organizer and coordinator of regular meeting. 
It is responsible for the following actions to make sure regular meeting 
is successful.

\subsubsection{call roles}

Call role means finding appropriate members to play each role in a meeting. 
Toastmaster must call for all meeting roles first. This is done in an official
way by sending an call role mail. The specification of an call role mail is shown
in section xxx.xxx. Pre-registered roles must be filled first. Other meeting roles
must use first come first served principle, and each role must be registered together
with a CC project(for speaker roles) or CL project(for other roles). Each role has 
a set of coresponding CC/CL project in accordance to CC/CL manual. Toastmaster 
should collect speech title in advance for each speaker role. All meeting roles
should be settled down two days before regular meeting date. 

\subsubsection{select a theme}
Each regular meeting has a theme. Toastmaster is responsible for selecting a meeting
theme, and inform the TTM in a timely manner. A brief disscussion considering TTM's 
advice is encouraged. Determining the theme of the meeting should be finished one day 
before regular meeting date.

\subsubsection{reconfirm meeting role}
Toastmaster should reconfirm each registered meeting role's presence one day before 
regular meeting date. Should there be any emergency(any registered role is cancelled),
Toastmaster may ask support from VPE to find necessary backup speaker(s).

\subsubsection{make meeting agenda}
Toastmaster should make a meeting agenda following ETMC's agenda template one day before 
regular meeting.

\subsubsection{send invitation mail}
Toastmaster should send regular meeting invitation following ETMC's meeting invitation 
template one day before regular meeting.

\subsubsection{conduct meeting}
Toastmaster may prepare a welcome speech, a series of introduction of role takers and 
a number of transition remarks before the meeting starts. A collect of information for
these preparation from members is encouraged. Toastmaster must conduct meeting according
to agenda.

\subsubsection{gratitude and upload agenda}
After the meeting, toastmaster should show gratitude to all role takers and upload 
meeting agenda to ETMC folder. 

\subsection{Grammarian}

Grammarian is the grammar observer for the whole meeting, targeting at correcting improper 
use of words and sentences for speakers to help members speak correct English. This role 
taker shall finish the following actions to achieve its goal.

\subsubsection{duty explanation}

Grammarian should prepare a duty explanation of grammarian role before the meeting and 
deliver it on the meeting.

\subsubsection{prepare word of the day}

Grammarian should prepare word of the day--it should be a word that will help members 
increase their vocabulary, and a word that can be incorporated easily into everyday
conversation. It is suggested that grammarian does a thorough research on the word about
pronounciation and example sentence to make sure in the meeting all members receive  
clear and correct information about that word. In the meeting, grammarian should promote
using of this word by various methods.

\subsubsection{grammar report}

Grammarian should listen to each speaker carefully and try to catch the good and erroneous 
usage of word and grammar. At the end of the meeting, a report about the grammar performance
of the meeting shall be delivered on the stage.

\subsection{Ah Counter}

The main purpose of an Ah-Counter role is to help members eliminate improper crutch or pause 
filler in a speech. It is responsible for the following actions in the meeting.

\subsubsection{duty explanation}
Ah Counter should prepare a duty explanation of Ah Counter role before the meeting and 
deliver it on the meeting.

\subsubsection{Ah report}
Ah Counter should listen carefully during the meeting and try to count the crutch or 
pause filler for each speaker. At the end of the meeting, a report about Ah 
performance of the meeting shall be delivered on the stage.

\subsection{Timer}
Timer controls time of various sessions in the meeting by showing speaker different 
signals about thier usage of time. By indicating how much time each speaker uses, 
Timer help members improve time management skills in the long run. The following 
actions should be done by Timer in a meeting.

\subsubsection{introduce time rule}
When on the stage, Timer should introduce about the time rule of the meeting and 
the meaning of different signal devices used to control meeting time. The following
is the time rule ETMC. If a speaker has less than or equal to 4 minutes time on stage, 
when 1 minute left, the green signal should be shown;when 30 seconds left, the yellow
signal should be shown; when time is used out, the red signal should be shown. If a 
speak has more than 4 minutes time on stage, when 2 minutes is left, the green signal
should be shown; when 1 minutes is left, the yellow signal should be shown, when time
is used out, the red signal should be shown. In either case, when red signal is shown,
speaker should finish speech within 30 seconds, otherwise Timer should knock table to
force the speaker off the stage.

\subsubsection{Time report}
Timer should record each participant's name and time used. Those who used time very 
well and very badly, should be mentioned in the time report at the end of the meeting.
After meeting, Timer should send out detailed time report to all members.

\subsection{Table Topic Master}
Table Topic session in ETMC creates an opportunity, for those who do not have a 
meeting role, to speak during the meeting. Meanwhile, it also help members think fast 
and practice delivering impromptu speeches. Table Topic Master is the host of this 
session, and the following actions should be done.

\subsubsection{prepare questions}
Table Topic Master should prepare several questions that surround a main theme before 
the meeting. The originality of these questions is desired as much as possible, and 
will be used in table topic session of the meeting. The questions should be short and
simple, and is likely to inspire people to speak.

\subsubsection{host table topics}
Table Topic session has a time length of 15 minutes, during which Table Topic Master 
should try to encourage more people who has not registered any role in the meeting to
speak. Table Topic Master should give a brief introduce of the purpose and the theme 
of the session within less than 3 minutes to the majority of time should be used by 
other speakers. When called on the stage, each speaker may have 20 seconds to think 
about the question and 2 minutes to deliver a mini impromptu speech. 

\subsection{Speech Evaluator}
Evaluator help improve members' speaking and leadership skills by giving evaluation 
according to speakers CC or CL project. The purpose of evaluation is to help speaker
become less self-conscious, more confident and more effective better speaker. 
Evaluator is responsible for the following actions.

\subsubsection{prepare evaluation}
Before the meeting, Speech Evaluator should know which CC project the speaker is 
practicing and what are the project goals. If the speaker to be evaluated has specific
evaluation request, Speech Evaluator should take it into consideration when giving
evaluation. This role must study carefully Effective Evaluation manual before doing 
evaluation.

\subsubsection{conduct evaluation}
Speech Evaluator should look at the speaker, and give an evaluation based on the 
speaker's CC project. The evalution should be mainly courageous and praising, leaving
speaker with specific methods for improving. 

\subsection{Table Topic Evaluator}
\subsection{Speaker}
\subsection{General Evaluator}

\section{Revision history}
\section{Reference}

%%%%%%%%%%%%%%%%%%%%%%%%%%%%%%%%%%%%%%%%%%%%%%%%%%%%%%%%%%%%%%%%%%%%%%%%%%
\section{call role mail}
\subsection{mail header}
Call role mail has a title, a to list,a cc list and its contents. The title has a specific 
format ``[ETMC]Call roles for regular meeting on <date>'', in which <date> must be replaced
by the exact date on which meeting will be held. The to list shall be all currently registered 
ETMC members. The cc list shall be left blank.

Mail contents shall have four sections. Section one is words to encourage members to 
take roles. Section two is meeting registration table which clearly shows role name, related 
CL projects, role owner and speech rehearsal person if this role is a speaker. Section three 
is a table with all CL project number and project name shown to remind role takers of practicing
CL projects while taking roles. Section four is promotion of role pre-register plan to encourage
members to register roles in future.



\bye
